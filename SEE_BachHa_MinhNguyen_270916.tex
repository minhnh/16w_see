\documentclass[a4paper, 12pt]{article}
\usepackage{titling}
\usepackage{array}
\usepackage{booktabs}
\usepackage{graphicx}
\setlength{\heavyrulewidth}{1.5pt}
\setlength{\abovetopsep}{4pt}

\setlength{\droptitle}{-7em}

\title{Scientific Experimentation and Evaluation \\
				- Homework 1 -}
\author{Bach Ha, Minh Nguyen}
\date{Lecture date: 27 September 2016}

\begin{document}

\maketitle


\begin{enumerate}
    \item \textbf{Experiment design:}
    \begin{itemize}
        \item Measurand: linear and angular displacement
        \item Measurement facility: ruler and protractor, white sheet of paper for marking poses and locations.
        \item DUT: Nxt robot assembled as instructed
        \item Measuring principle: linear comparison
        \item Measuring method:
            \begin{itemize}
                \item To limit amount of alignment error and external influences, start pose and end pose are measured with an rectangle ``enclosure'' which just barely fits the robot. The enclosure will have markings aligned with the x-axis (front and back) of the robot. Using the markings on the enclosure can minimize errors from alignment, and ensuring the robot is not moved when the enclosure is fitted on at start and end pose can reduce external influences on the measurement.
                \item Displacement will be measured between midpoints of the lines connecting the front and back markings at start and end pose. This line also represents the relative pose of the start and end point.
                \item At the beginning of the measurement, the robot is placed within the enclosure aligned to a predefined starting location.
                \item After removing the enclosure and executing the movement command, the enclosure is carefully placed back on around the robot again, and the marking of the end pose will be noted.
                \item The distance between the midpoints and the change in angles between the lines connecting the markings between the poses respectively represent the linear and angular displacements of the movement.
            \end{itemize}
    \end{itemize}

    \item \textbf{Potential error source:}
    \begin{itemize}
        \item The starting placement.
        \item Drawing and measuring of lines and midpoints.
        \item Measurement of angles between lines.
        \item Rounding of calculation from lengths and angles to x,y coordination.
        \item Error of the measurement facility.
    \end{itemize}

\end{enumerate}



\end{document}
