\documentclass[a4paper, 12pt]{article}
\usepackage{titling}
\usepackage{array}
\usepackage{booktabs}
\usepackage{graphicx}
\setlength{\heavyrulewidth}{1.5pt}
\setlength{\abovetopsep}{4pt}

\setlength{\droptitle}{-7em}

\title{Scientific Experimentation and Evaluation \\
				- Homework 1 -}
\author{Bach Ha, Minh Nguyen}
\date{Lecture date: 27 September 2016}

\begin{document}

\maketitle


\begin{enumerate}
    \item \textbf{Experiment design:}
    \begin{itemize}
        \item Measurand: linear and angular displacement
        \item Measurement facility: ruler and protractor, white sheet of paper for marking poses and locations.
        \item DUT: Nxt robot assembled as instructed
        \item Measuring principle: linear comparison
        \item Measuring method:
            \begin{itemize}
                \item Robot's start point and end point will be marked each with two parallel lines. One line behind the rear wheels, which is parallel to the wheel axle. The second line is in front of the front wheel and is parallel to the first line.
                \item The robot center point is the center of the line P draw from the middle of the rear wheel axle, and is perpendicular to the first two lines.
                \item That line also represent the robot's orientation.
                \item After each run, the start center point and stop center point are connected to make a line L.
                \item The x axis and y axis will be parallel with one of the white paper's edge.
                \item The robot, at the starting point, will be placed parallel to one of the paper's edge and will always has the pose (0,0,0)
                \item The length of line L and the angle between L and the x axis will be measured to calculate the robot's x,y coordinate.
                \item The robot's orientation is the angle between line P of the stop and the x axis.
            \end{itemize}
    \end{itemize}

    \item \textbf{Potential error source:}
    \begin{itemize}
        \item The starting placement.
        \item Drawing and measuring of line L and P.
        \item Measurement of angle between line L or P and the x axis.
        \item Rounding of calculation from length and angle to x,y coordination.
        \item Error of the measurement facility.
    \end{itemize}

\end{enumerate}



\end{document}
