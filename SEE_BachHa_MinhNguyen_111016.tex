\documentclass[paper=a4, fontsize=11pt]{scrartcl} % KOMA-article class

%-------------------------------------------------
%   THEMES, PACKAGES, CUSTOM COMMANDS
%-------------------------------------------------
\usepackage{blindtext}
\usepackage[english]{babel}                             % English language/hyphenation
\usepackage[protrusion=true,expansion=true]{microtype}  % Better typography
\usepackage{amsmath,amsfonts,amsthm}                    % Math packages
\usepackage[pdftex]{graphicx}                           % Enable pdflatex
\usepackage[export]{adjustbox}
\usepackage[svgnames]{xcolor}                           % Enabling colors by their 'svgnames'
\usepackage[hang, small,labelfont=bf,up,textfont=it,up]{caption} % Custom captions under/above floats
\usepackage{subcaption}
\usepackage{epstopdf}       % Converts .eps to .pdf
%\usepackage{subfig}         % Subfigures
\usepackage{booktabs}       % Nicer tables
\usepackage{fix-cm}         % Custom fontsizes
\usepackage{listings}
\usepackage{soul}
\usepackage{float}

\usepackage{hyperref}

\usepackage[foot=30pt,margin=1in]{geometry}

% Custom sectioning (sectsty package)
\usepackage{sectsty}
\allsectionsfont{
    \usefont{OT1}{phv}{b}{n}    % bch-b-n: CharterBT-Bold font
}
\sectionfont{
    \usefont{OT1}{phv}{b}{n}
}

% Custom colors
\definecolor{brsugrey}{rgb}{0.9, 0.9, 0.9}
\definecolor{brsublue}{rgb}{0, 0.594, 0.949}

%
\newcommand{\upperRomannumeral}[1]{\uppercase\expandafter{\romannumeral#1}}

% Creating an initial of the very first character of the content
\usepackage{lettrine}
\newcommand{\initial}[1]{%
    \lettrine[lines=3,lhang=0.3,nindent=0em]{
        \color{brsublue}
        {\textsf{#1}}}{}}

%-------------------------------------------------
%   COMMON INFO
%-------------------------------------------------
\newcommand{\hmwkTitle}{Week \upperRomannumeral{3} report}
\newcommand{\hmwkDueDate}{Tuesday, October 11, 2016}
\newcommand{\hmwkClass}{Scientific Evaluation and Experimentation}
\newcommand{\hmwkClassShort}{SEE WS2016}
\newcommand{\hmwkAuthorFullName}{Minh H. Nguyen \& Bach D. Ha}
\newcommand{\hmwkAuthorLastName}{Nguyen \& Ha}
\newcommand{\hmwkAuthorEmail}{minh.nguyen@smail.inf.h-brs.de\\
                              bach.ha@smail.inf.h-brs.de}
\newcommand{\hmwkAuthorInstitute}{BRS University of Applied Sciences}

%-------------------------------------------------
%   HEADERS & FOOTERS
%-------------------------------------------------
\usepackage{fancyhdr}
\pagestyle{fancy}
\usepackage{lastpage}
% Header (empty)
\lhead{}
\chead{}
\rhead{}
% Footer (you may change this to your own needs)
\lfoot{\footnotesize
    \texttt{\hmwkClassShort} ~
    \textbullet ~ \hmwkAuthorLastName ~
    \textbullet ~ \hmwkTitle}
\cfoot{}
\rfoot{\footnotesize page \thepage\ of \pageref{LastPage}}  % "Page 1 of 2"
\renewcommand{\headrulewidth}{0.0pt}
\renewcommand{\footrulewidth}{0.4pt}

%-------------------------------------------------
%   TITLE & AUTHOR
%-------------------------------------------------
\usepackage{titling}

\newcommand{\HorRule}{\color{brsublue}% Creating a horizontal rule
    \rule{\linewidth}{1pt}%
    \color{black}
}

% Title
\pretitle{
    \vspace{-30pt}
    \begin{flushleft}
        \HorRule
        \fontsize{22}{22} \usefont{OT1}{phv}{b}{n} \color{gray} \selectfont
}
\title{\hmwkClass \\
       \hmwkTitle}
\posttitle{
    \par
    \end{flushleft}
    \vskip 0.5em
}

% Author
\preauthor{
    \begin{flushleft}
        \large \lineskip 0.25em
        \usefont{OT1}{phv}{b}{sl} \color{brsublue}}

\author{\hmwkAuthorFullName}

\postauthor{
        \footnotesize
        \usefont{OT1}{phv}{m}{sl} \color{Black}
        \\\hmwkAuthorInstitute
        \\\hmwkAuthorEmail
        \par
    \end{flushleft}
    \HorRule}

% Date
\date{\hmwkDueDate}

%-------------------------------------------------
%   BEGIN
%-------------------------------------------------
\begin{document}
    \maketitle
    \thispagestyle{fancy} % Enabling the custom headers/footers for the first page

    \section{Camera calibration}

    \begin{figure}[h!]
        \centering
        \includegraphics[width=0.9\linewidth]{images/cameraCalib_sample.png}
        \caption{Image calibration using MATLAB}
        \label{fig:img1}
    \end{figure}

    \subsection{Calibration procedure}
    \begin{enumerate}
        \item The camera is mounted on a tripod at a fixed position.
        \item 40 pictures from different angles of the calibration chess board is taken.
        \item These 40 pictures are fed to Matlab's camera calibration toolbox.
        \item The side length of each square (72 mm) is provided to the toolbox.
        \item The calibration process is done automatically by the toolbox
    \end{enumerate}
    
    Note:
    \begin{itemize}
        \item Calibration chess board has size 9x7 squares.
        \item Side length of each square is 72 mm.
        \item Bad lighting condition can lead to long calibration calculation and inaccurate result.
        \item Calibration board should take most of the space in a picture.
        \item The facing angle of the board to the camera should not be too extreme, otherwise the calibration toolbox will not be able to recognize the board.
    \end{itemize}

    \subsection{Explanation of camera parameters}
    The calibration process finds the parameters of a pinhole model for the camera. In the pinhole camera model:
    \begin{itemize}
        \item Intrinsic matrix $K$ is parameterized by the geometric properties of the camera:
        \[ \left[ \begin{matrix}
        f_x    & s      & x_0    \\
        0      & f_y    & y_0    \\
        0      & 0      & 1
        \end{matrix} \right] \]
        \item Focal length $\left[ \begin{matrix} fx & fy \end{matrix} \right]$ is the distance between the focal point and the image plane.
        \item $s$ is the skew parameter, which results in shear distortion of the image.
        \item $\left[ \begin{matrix} x_0 & y_0 \end{matrix} \right]$ are coordinates of the principal point, or the offset of the principal point to the film's origin.
    \end{itemize}

    \subsection{Calibration results}
    \begin{itemize}
        \item Intrinsic camera matrix:
        \[ \left[ \begin{matrix}
        1451.495    & 0           & 969.512    \\
        0           & 1451.626    & 552.246    \\
        0           & 0           & 1
        \end{matrix} \right] \]

        \item Radial distortion: $\left[ \begin{matrix} 0.0012 & 0.0043 \end{matrix} \right]$

        \item Mean reprojection error: $0.263$
    \end{itemize}

    \newpage
    \section{Principal Component Analysis}

    \subsection{Going straight forward}
    \begin{figure}[h!]
        \begin{center}
            \setlength{\fboxsep}{0.5pt} %
            \setlength{\fboxrule}{0.5pt}
            \includegraphics[width=0.7\linewidth,fbox]{images/poses_plot_1_straight.png}
            \caption{Recorded poses of the LeJOS robot going straight forward.}
        \end{center}
    \end{figure}
    \begin{figure}[h!]
        \begin{center}
            \setlength{\fboxsep}{0.5pt} %
            \setlength{\fboxrule}{0.5pt}
            \includegraphics[width=\linewidth,fbox]{images/pca_straight.png}
            \caption{PCA analysis of straight forward movement.}
        \end{center}
    \end{figure}
    \begin{figure}[h!]
        \begin{center}
            \setlength{\fboxsep}{0.5pt} %
            \setlength{\fboxrule}{0.5pt}
            \includegraphics[width=\linewidth,fbox]{images/pca_histogram_straight.png}
            \caption{Histograms of coordinates projected onto principal components for straight forward movement.}
        \end{center}
    \end{figure}

    \subsection{Going slightly left}
    \begin{figure}[h!]
        \begin{center}
            \setlength{\fboxsep}{0.5pt} %
            \setlength{\fboxrule}{0.5pt}
            \includegraphics[width=0.7\linewidth,fbox]{images/poses_plot_2_slightLeft.png}
            \caption{Recorded poses of the LeJOS robot going slightly left.}
        \end{center}
    \end{figure}

    \begin{figure}[h!]
        \begin{center}
            \setlength{\fboxsep}{0.5pt} %
            \setlength{\fboxrule}{0.5pt}
            \includegraphics[width=\linewidth,fbox]{images/pca_slightLeft.png}
            \caption{PCA analysis of slight left movements.}
        \end{center}
    \end{figure}

    \begin{figure}[h!]
        \begin{center}
            \setlength{\fboxsep}{0.5pt} %
            \setlength{\fboxrule}{0.5pt}
            \includegraphics[width=\linewidth,fbox]{images/pca_histogram_slightLeft.png}
            \caption{Histograms of coordinates projected onto principal components for slight left movements.}
        \end{center}
    \end{figure}

    \newpage
    \subsection{Going slightly right}
    \begin{figure}[H]
        \begin{center}
            \setlength{\fboxsep}{0.5pt} %
            \setlength{\fboxrule}{0.5pt}
            \includegraphics[width=12cm,fbox]{images/poses_plot_3_slightRight.png}
            \caption{Recorded poses of the LeJOS robot going slightly right.}
        \end{center}
    \end{figure}

    \begin{figure}[h!]
        \begin{center}
            \setlength{\fboxsep}{0.5pt} %
            \setlength{\fboxrule}{0.5pt}
            \includegraphics[width=\linewidth,fbox]{images/pca_slightRight.png}
            \caption{PCA analysis of slight right movements.}
        \end{center}
    \end{figure}

    \begin{figure}[h!]
        \begin{center}
            \setlength{\fboxsep}{0.5pt} %
            \setlength{\fboxrule}{0.5pt}
            \includegraphics[width=\linewidth,fbox]{images/pca_histogram_slightRight.png}
            \caption{Histograms of coordinates projected onto principal components for slight right movements.}
        \end{center}
    \end{figure}

    \newpage
    \subsection{Going left}
    \begin{figure}[h!]
        \begin{center}
            \setlength{\fboxsep}{0.5pt} %
            \setlength{\fboxrule}{0.5pt}
            \includegraphics[width=12cm,fbox]{images/poses_plot_4_left.png}
            \caption{Recorded poses of the LeJOS robot going left.}
        \end{center}
    \end{figure}

    \begin{figure}[h!]
        \begin{center}
            \setlength{\fboxsep}{0.5pt} %
            \setlength{\fboxrule}{0.5pt}
            \includegraphics[width=\linewidth,fbox]{images/pca_Left.png}
            \caption{PCA analysis of hard left movements.}
        \end{center}
    \end{figure}
    
    \begin{figure}[h!]
        \begin{center}
            \setlength{\fboxsep}{0.5pt} %
            \setlength{\fboxrule}{0.5pt}
            \includegraphics[width=\linewidth,fbox]{images/pca_histogram_Left.png}
            \caption{Histograms of coordinates projected onto principal components for hard left movements.}
        \end{center}
    \end{figure}

    \newpage
    \subsection{Going right}
    \begin{figure}[h!]
        \begin{center}
            \setlength{\fboxsep}{0.5pt} %
            \setlength{\fboxrule}{0.5pt}
            \includegraphics[width=12cm,fbox]{images/poses_plot_5_right.png}
            \caption{Recorded poses of the LeJOS robot going right.}
        \end{center}
    \end{figure}

    \begin{figure}[h!]
        \begin{center}
            \setlength{\fboxsep}{0.5pt} %
            \setlength{\fboxrule}{0.5pt}
            \includegraphics[width=\linewidth,fbox]{images/pca_Right.png}
            \caption{PCA analysis of hard right movements.}
        \end{center}
    \end{figure}
    
    \begin{figure}[h!]
        \begin{center}
            \setlength{\fboxsep}{0.5pt} %
            \setlength{\fboxrule}{0.5pt}
            \includegraphics[width=\linewidth,fbox]{images/pca_histogram_Right.png}
            \caption{Histograms of coordinates projected onto principal components for hard right movements.}
        \end{center}
    \end{figure}

\end{document}
